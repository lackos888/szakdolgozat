\pagestyle{empty}

\noindent \textbf{\Large Adathordozó használati útmutató}

\vskip 1cm

Az adathordozó felépítése egyszerű. Tartalmazza a következő jegyzékeket és fájlokat:
\begin{itemize}
	\item \textbf{latex} jegyzék: Ebben található maga a dolgozat \LaTeX fájljai, benne a fejezetekkel és a mellékletekkel (képekkel).
	\item \textbf{prog} jegyzék: Ez tartalmazza magát a programot. A prog jegyzék felépítése:
		\begin{itemize}
			\item \textbf{\detokenize{openvpn}} jegyzék: Ebben található a saját Certificate Authoritynk (Easy-RSA), ami mind a kliensek, és a szerver certificate \detokenize{&} private key kezeléséért felelős. Runtime hozódik létre, a program használata során.
			\item \textbf{\detokenize{easy_rsa_install_cache}} jegyzék: Runtime hozódik létre, az EasyRSA archívuma található benne.
			\item \textbf{\detokenize{modules}} jegyzék: Itt található maga a program moduljai .lua fájlokban, továbbá ezeken belül is vannak modulokat tartalmazó aljegyzékek.
			\item \textbf{\detokenize{authenticator.lua}} fájl: Certbot használatához szükséges fájl, a DNS-01 challenge implementálásáért felelős.
			\item \textbf{\detokenize{authenticator.sh}} fájl: Egyszerű shell script, amely meghívja az \\\texttt{authenticator.lua} fájlt a Lua interpreterével.
			\item \textbf{\detokenize{main.lua}} fájl: Ez a fájl tartalmazza maga a program kezelőfelületét, továbbá a program implementálása során felmerülő tesztelési kódokat.
		\end{itemize}
	\item \textbf{dolgozat.pdf} fájl: Az elkészült dolgozat PDF formátumban.
\end{itemize}