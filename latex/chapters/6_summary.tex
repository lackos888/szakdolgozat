\Chapter{Összefoglalás}

A dolgozat célja volt, hogy egy olyan Lua nyelvben általam készített eszközt mutasson be és készítsen el, amely könnyebbé teszi GNU/Linux disztribúciókon a szerverüzemeltetést.

Úgy gondolom, hogy ezt a célt sikeresen teljesítettem. Sikerült egy olyan eszköztárat létrehoznom, amely a mindennapokban megkönnyítheti a szerverüzemeltetést azzal, hogy néhány műveletet átvesz a felhasználótól, például: webszerverek kezelését, tűzfal kezelését, OpenVPN szerver kezelését. Véleményem szerint az elkészült implementáció a gyakorlatban is megállja a helyét, éles környezetben, valós feladatok ellátására is használható lenne.

A megvalósítás során sok érdekes (és néhányszor akadályozó) dologgal találkoztam, és habár nem szakdolgozatom során találkoztam először a Lua programozási nyelvvel és a GNU/Linux alapú Linux disztribúciókkal, úgy érzem sikerült a szakdolgozat megírása során a már meglévő tudásomat bővíteni.

Igyekeztem a program implementálása során minőségi és könnyen átlátható kódot létrehozni, amely későbbi továbbfejlesztési lehetőséget hordozhat magában.

A tervekhez képest a funkcionalitás kissé elmaradt, azonban úgy gondolom, hogy így is jól használható az elkészült alkalmazás. Lehetne még tovább fejleszteni a programot, például:
\begin{itemize}
	\item Command line argumentumok létrehozásával, hogy támogassa az automatizációt. Ennek megfelelően különböző process exit codekra lenne szükség
	\item A kezelőfelülethez lehetne Terminal UI támogatást létrehozni, hogy méginteraktívabb legyen a program, ezzel például támogathatná az egérkattintásokat is, és az egyszerűbb navigációt
	\item További szerverek támogatása
	\item Automatikus logértelmezés, olvasás (például connection log, iptables block log, access log)
	\item OpenVPN webadmin felület implementálásával
	\item Teljeskörű webadmin felület implementálásával
	\item Több Linux disztribúció támogatásával (különböző csomagkezelő programok támogatása)
	\item Esetleges Windows OS támogatással. Habár maga az implementáció egyelőre csak Linux disztribúciókhoz készült el, a programozási nyelvnek köszönhetően az elkészült szoftver platformfüggetlen lehetne. A jövőben akár ugyanebben a kódbázisban teljesen más platformok támogatását könnyedén lehetővé lehetne tenni.
\end{itemize}

Úgy gondolom, a felsorolt továbbfejlesztési lehetőségekkel való kibővítéssel akár a jelenleg elérhető népszerű eszközök alternatívája is lehet.