\Chapter{Tervezés}

A program az OO alapelveket igyekszik követni. Elsősorban modulokból áll, amelyek osztályként is értelmezhetőek. Vannak olyan modulok, amelyekben csak a helyi scopeban elérhető kódok vannak, tehát "private" funkciók. Továbbá vannak olyan modulok, amelyekben szintén vannak további osztályok.

A következőekben a modulok UML diagramját fogom külön-külön bemutatni.
\begin{center}
	Következőképp néznek ki a láthatósági szabályok az UML diagramokon:
	
	\hfill
	
	\begin{tikzpicture} % [show background grid]
		\begin {class}[text width=0.35\textwidth]{Láthatósági szabályok}{0, 0}
			\attribute {+ Public}
			\attribute {- Private}
		\end {class}
	\end {tikzpicture}
\end{center}

\Section{general, apt\_packages, utils modulok felépítése, feladata}

Legelőször a \detokenize{general, linux, apt_packages} és az utils modul került megtervezésre.

A \texttt{general} modulban olyan funkciók és változók találhatóak, amelyek nem OS-specifikusak, bármelyik OS-en működnek, és fontos szerepet töltenek be a program működése közben.

Az \texttt{\detokenize{apt_packages}} modulban Linux-specifikus kód található, a \texttt{linux} modulban implementált néhány funkciót használja. Feladata az apt program segítségével a csomagok menedzselése, telepítése.

\begin{figure}[h]
	\centering
	% \caption{A general és \detokenize{apt_packages} modul felépítése}
		\begin{tikzpicture}
			\begin {class}[text width=0.46\textwidth]{general}{0, 0}
				\attribute {+ \detokenize{lineEnding : string}}
				\operation {+ \detokenize{getOSType()}}
				\operation {+ \detokenize{strSplit(str, sep)}}
				\operation {+ \detokenize{deep_compare(tbl1, tbl2)}}
				\operation {+ \detokenize{concatPaths(...) -> varargs}}
				\operation {+ \detokenize{extractDirFromPath(path)}}
				\operation {+ \detokenize{readAllFileContents(filePath)}}
			\end {class}
			\begin {class}[text width=0.44\textwidth]{\detokenize{apt_packages}}{8, 0}
				\operation {+ \detokenize{is_package_installed(packageName)}}
				\operation {+ \detokenize{install_package(packageName)}}
			\end {class}
		\end {tikzpicture}
\end{figure}

\pagebreak

\begin{center}
	Az \texttt{utils} modul legfőképp debugra volt használva, jelenleg épp nincs alkalmazva a program kódjában:
	
	\hfill
	
	\begin{tikzpicture}
		\begin {class}[text width=0.5\textwidth]{utils}{0, 0}
			\attribute {+ \detokenize{tprint(tbl, indent)}}
		\end {class}
	\end {tikzpicture}
\end{center}

\Section{linux modul felépítése, feladatai}

\begin{center}
	A következő UML diagramon a \texttt{linux} nevezetű modult fogom bemutatni, amely linux-specifikus kódrészleteket tartalmaz. Erre a modulra épül a legtöbb modul.
	
	\hfill
	
	\begin{tikzpicture} % [show background grid]
		\begin {class}[text width=0.43\textwidth]{linux}{0, 0}
			\operation {+ \detokenize{exists(file)}}
			\operation {+ \detokenize{isdir(path)}}
			\operation {+ \detokenize{listDirFiles(path)}}
			\operation {+ \detokenize{mkdir(path)}}
			\operation {+ \detokenize{deleteFile(path))}}
			\operation {+ \detokenize{deleteDirectory(path)}}
			\operation {+ \detokenize{exec_command(cmd)}}
			\operation {+ \detokenize{check_if_user_exists(userName)}}
			\operation {+ \detokenize{create_user_with_name(userName, comment, shell, homeDir)}}
			\operation {+ \detokenize{update_user(userName, comment, shell)}}
			\operation {+ \detokenize{get_user_home_dir(userName)}}
			\operation {+ \detokenize{exec_com-}\\\detokenize{mand_with_proc_ret_code(}\\cmd, linesReturned, envVariables, redirectStdErrToStdIn\detokenize{)}}
			\operation {+ \detokenize{copy(from, to)}}
			\operation {+ \detokenize{copyAndChown(user, from, to)}}
			\operation {+ \detokenize{chown(path, userName, isDir)}}
			\operation {+ \detokenize{chmod(path, perm, isDir)}}
		\end {class}
	\end {tikzpicture}
\end{center}

A \texttt{linux} modul feladatai szerteágazóak:

\begin{itemize}
	\item vannak benne fájl- és könyvtár manipulációs funkciók
	\item vannak benne parancsfuttató funkciók (amelyek közül az egyik tökéletesen kompatibilis Windows-sal is)
	\item vannak benne felhasználókat módosító funkciók (felhasználók létrehozása, módosítása, felhasználó létezésének ellenőrzése, felhasználó home dir lekérése)
	\item vannak benne ownershipet változtató funkciók
	\item vannak benne fájl és mappa hozzáférési szabályokat változtató funkciók
\end{itemize}

\pagebreak

\Section{OpenVPN modulok felépítései, feladatai}

A következőekben az OpenVPN szerver kezelését kezdtem el megtervezni, implementálni. Az összes modul a program jegyzékén belül a \textbf{modules/vpnHandler} jegyzékben található meg.

Több részmodulra lett szétosztva:
\begin{itemize}
	\item \textbf{OpenVPN}: ez maga \texttt{bootstrap} modul, ebben van OpenVPN csomagot feltelepítő funkció, továbbá ez a modul tölti be a \texttt{\detokenize{server_impl}} modult
	\item \textbf{\detokenize{server_impl}}: ez a modul kezeli a szerverrel kapcsolatos legtöbb dolgot
	
	előkészíti a könyvtárakat, feltelepíti az Easy-RSA Certificate kezelőt, létrehoz egy saját CA-t (\texttt{Certificate Authority}-t), szerver-oldali certificate és kulcsfájlokat generál, saját OpenVPN usert hoz létre a szervernek, tls-auth/tls-crypt kulcsot generál, a server configot beüzemeli, beállítja a server daemon auto-startot a \texttt{/etc/default/openvpn} fájlban

	\item \textbf{\detokenize{config_handler}}: az OpenVPN szerver és kliens konfigját kezelő modul, config parselést és writeolást implementál. A parser és writer az eredeti OpenVPN parser kódja alapján épült, amely megtalálható a \cite{openvpn_parser} hivatkozás alatt.
	\item \textbf{\detokenize{clienthandler_impl}}: ez a modul kezel néhány dolgot a kliensekkel kapcsolatosan.
	
	kliens certificatet, private keyt hoz létre; kliens konfigot hoz létre, amelybe beembeddeli a generált fájlok tartalmát (így 1 db config filera van szüksége az OpenVPN kliensnek); certificate revoket támogat; továbbá le lehet kérni az összes jelenlegi klienst, amelyek valósak (nincsenek revokeolva)
\end{itemize}

\begin{center}
	Maga az OpenVPN \texttt{bootstrap} modul nagyon egyszerű felépítésű:
	
	\hfill
	
	\begin{tikzpicture} % [show background grid]
		\begin {class}[text width=0.45\textwidth]{OpenVPN}{0, 0}
			\attribute {+ \detokenize{server_impl -> }\textbf{OpenVPN\_server\_impl} module}
			\operation {+ \detokenize{is_openvpn_installed()}}
			\operation {+ \detokenize{install_openvpn()}}
		\end {class}
	\end {tikzpicture}
	
	\hfill
	
	Hozzá hasonlóan a \texttt{\detokenize{config_handler}} modul is egyszerűen épül fel:
	
	\hfill
	
	\begin{tikzpicture} % [show background grid]
		\begin {class}[text width=0.45\textwidth]{\detokenize{OpenVPN_config_handler}}{0, 0}
			\attribute {- \detokenize{general -> }\textbf{general} module}
			\operation {+ \detokenize{parse_openvpn_config(linesInStr)}}
			\operation {+ \detokenize{write_openvpn_config(parsedLines)}}
		\end {class}
	\end {tikzpicture}
\end{center}

\pagebreak

\begin{center}
	A \texttt{\detokenize{clienthandler_impl}} modul már bonyolultabb felépítésű, két osztályt is tartalmaz:
	
	\begin{figure}[h]
		\begin{tikzpicture} % [show background grid]
			\begin {class}[text width=0.46\textwidth]{\detokenize{OpenVPN_clienthandler_impl}}{0, 0}
				\attribute {- \detokenize{linux -> }\textbf{\detokenize{linux}} module}
				\attribute {- \detokenize{general -> }\textbf{\detokenize{general}} module}
				\attribute {- \detokenize{utils -> }\textbf{\detokenize{utils}} module}
				\attribute {- \detokenize{config_handler -> }\textbf{\detokenize{OpenVPN_config_handler}} module}
				\attribute {- \detokenize{server_impl -> }\textbf{\detokenize{OpenVPN_server_impl}} module}
				\attribute {- \detokenize{clientObjects : table}}
				\attribute {- \detokenize{validClients : table}}
				\attribute {- \detokenize{clientSampleConfig : string}}
				\attribute {+ \detokenize{Client -> }\textbf{class}}
				\operation {- \detokenize{get}\\ \detokenize{_valid_clients_from_PKI_database()}}
				\operation {+ \detokenize{constructor(openVPNServerImpl)}}
				\operation {+ \detokenize{update}\\ \detokenize{_revoke_crl_for_openvpn_daemon()}}
				\operation {+ \detokenize{get_valid_clients()}}
			\end {class}
			\begin {class}[text width=0.44\textwidth]{\detokenize{Client}}{8.25, 0}
				\attribute {- \detokenize{linux -> }\textbf{\detokenize{linux}} module}
				\attribute {- \detokenize{general -> }\textbf{\detokenize{general}} module}
				\attribute {- \detokenize{utils -> }\textbf{\detokenize{utils}} module}
				\attribute {- \detokenize{config_handler -> }\textbf{\detokenize{OpenVPN_config_handler}} module}
				\attribute {- \detokenize{server_impl -> }\textbf{\detokenize{OpenVPN_server_impl}} module}
				\attribute {- \detokenize{clientObjects : table}}
				\attribute {- \detokenize{validClients : table}}
				\attribute {- \detokenize{clientSampleConfig : string}}
				\operation {+ \detokenize{new(clientName, loadedFromPKI)}}
				\operation {+ \detokenize{genKeyAndCRT(password)}}
				\operation {+ \detokenize{generateClientConfig()}}
				\operation {+ \detokenize{revoke()}}
				\operation {+ \detokenize{isValidClient()}}	
			\end {class}
			
			\unidirectionalAssociation{OpenVPN_clienthandler_impl}{}{}{Client}
		\end {tikzpicture}
	\end{figure}
\end{center}

\Section{nginx modulok felépítései, feladatai}

Az nginx-et kezelő modulok is több részre lettek osztva. Az összes modul a program jegyzékén belül a \textbf{modules/nginxHandler} jegyzékben található meg.

A modulok leírásai, feladatokkal:
\begin{itemize}
	\item \textbf{nginx}: ez maga egy \texttt{bootstrap} modul, ebben van nginx csomagot feltelepítő funkció, továbbá ez a modul tölti be a \texttt{\detokenize{server_impl}} modult
	\item \textbf{\detokenize{server_impl}}: ez a modul kezeli a szerverrel kapcsolatos legtöbb dolgot
	
	előkészíti a könyvtárakat, létrehoz az nginx daemonnak, workereknek egy usert; támogatja automatikusan a weboldalak létrehozását, törlését; támogatja az SSL-t; minden támogatott featuret bekonfigurál automatikusan

	\item \textbf{\detokenize{config_handler}}: nginx szerver konfigját kezelő modul, config parselést és writeolást implementál. A parser és writer az eredeti nginx parser kódja alapján épült, amely megtalálható a \cite{nginx_parser} hivatkozás alatt
\end{itemize}

\begin{center}
	Az nginx \texttt{bootstrap} modul ezesetben is egyszerű felépítésű:
	
	\hfill
	
	\begin{tikzpicture} % [show background grid]
		\begin {class}[text width=0.45\textwidth]{nginx}{0, 0}
			\attribute {+ \detokenize{server_impl -> }\textbf{nginx\_server\_impl} module}
			\operation {+ \detokenize{is_nginx_installed()}}
			\operation {+ \detokenize{install_nginx()}}
			\operation {+ \detokenize{init_dirs()}}
		\end {class}
	\end {tikzpicture}
\end{center}
	
\pagebreak

Azonban a \texttt{\detokenize{config_handler}} modul ebben az esetben már kissé bonyolultabb, mivel az nginx syntaxa is komplikáltabb. Két osztályt tartalmaz. Az ábrán látható a \texttt{\detokenize{nginx_server_impl}} modul felépítése is.

\hfill

\begin{tikzpicture} % [show background grid]
	\begin {class}[text width=0.45\textwidth]{\detokenize{nginx_config_handler}}{0, 0}
		\attribute {- \detokenize{CR : string}}
		\attribute {- \detokenize{LF : string}}
		\attribute {- \detokenize{NGX_ERROR : number}}
		\attribute {- \detokenize{NGX_CONF_BLOCK_DONE : number}}
		\attribute {- \detokenize{NGX_CONF_FILE_DONE : number}}
		\attribute {- \detokenize{NGX_CONF_BLOCK_START : number}}
		\attribute {- \detokenize{NGX_CONF_BLOCK_DONE : number}}
		\attribute {- \detokenize{NGX_OK : number}}
		\attribute {- \detokenize{NGX_CONF_MAX_ARGS : number}}
		\attribute {- \detokenize{inspect -> }\textbf{inspect} module}
		\attribute {- \detokenize{general -> }\textbf{general} module}
		\attribute {+ \detokenize{nginxConfigHandler -> }\textbf{nginxConfigHandler} class}
		\operation {- \detokenize{concatArgsProperlyForBlockName(args)}}
		\operation {- \detokenize{parse_nginx_config(linesInStr)}}
		\operation {- \detokenize{formatDataAccordingQuoting(tbl)}}
		\operation {- \detokenize{doPaddingWithBlockDeepness(blockDeepness)}}
		\operation {- \detokenize{write_nginx_config(parsedLines)}}
	\end {class}

	\begin {class}[text width=0.44\textwidth]{\detokenize{nginxConfigHandler}}{7.5, 0}
		\operation {+ \detokenize{new(linesInStr, paramToLine)}}
		\operation {+ \detokenize{getParsedLines()}}
		\operation {+ \detokenize{getParamsToIdx()}}
		\operation {+ \detokenize{insertNewData(dataTbl, pos)}}
		\operation {+ \detokenize{deleteData(pos)} - jelenleg nem használt}
		\operation {+ \detokenize{toString()}}
	\end {class}
	
	\begin {class}[text width=0.45\textwidth]{\detokenize{nginx_server_impl}}{7.5, -5}
		\attribute {- \detokenize{os -> }\textbf{os} module}
		\attribute {- \detokenize{packageManager -> }\textbf{\detokenize{apt_packages}} module}
		\attribute {- \detokenize{linux -> }\textbf{linux} module}
		\attribute {- \detokenize{general -> }\textbf{general} module}
		\attribute {- \detokenize{nginxConfigHandlerModule -> }\textbf{\detokenize{config_handler}} module}
		\attribute {- \detokenize{inspect -> }\textbf{inspect} module}
		\attribute {- \detokenize{sampleConfigForWebsite : string}}
		\attribute {+ \detokenize{nginx_user : string}}
		\attribute {+ \detokenize{nginx_user_comment : string}}
		\attribute {+ \detokenize{nginx_user_shell : string}}
		\attribute {+ \detokenize{base_dir : string}}
		\operation {+ \detokenize{formatPathInsideBasedir(path)}}
		\operation {+ \detokenize{init_dirs()}}
		\operation {+ \detokenize{check_nginx_user_existence()}}
		\operation {+ \detokenize{create_nginx_user(homeDir)}}
		\operation {+ \detokenize{update_existing_nginx_user()}}
		\operation {+ \detokenize{get_nginx_home_dir()}}
		\operation {+ \detokenize{get_nginx_master_config_}\\\detokenize{path_from_daemon()}}
		\operation {+ \detokenize{initialize_server()}}
		\operation {+ \detokenize{create_new_website(websiteUrl)}}
		\operation {+ \detokenize{delete_website(websiteUrl)}}
		\operation {+ \detokenize{get_current_available_websites()}}
		\operation {+ \detokenize{init_ssl_for_website(webUrl, certDetails)}}
	\end {class}
	
	\unidirectionalAssociation{nginx_config_handler}{}{}{nginxConfigHandler}
\end {tikzpicture}

\Section{apache modulok felépítései, feladatai}

Az apache kezeléséhez tartozó modulok a program jegyzékén belül a \textbf{modules/apacheHandler} jegyzékben található meg.

A modulok leírásai, feladatokkal:
\begin{itemize}
	\item \textbf{apache}: ez maga egy \texttt{bootstrap} modul, ebben van apache2 csomagot feltelepítő funkció, továbbá ez a modul tölti be a \texttt{\detokenize{server_impl}} modult
	\item \textbf{\detokenize{server_impl}}: ez a modul kezeli a szerverrel kapcsolatos legtöbb dolgot
	
	előkészíti a könyvtárakat, létrehoz az apache2 daemonnak, processeknek egy usert\\támogatja automatikusan a weboldalak létrehozását, törlését; támogatja az SSL-t; minden támogatott featuret bekonfigurál automatikusan (envvars-t is)

	\item \textbf{\detokenize{config_handler}}: apache szerver konfigját kezelő modul\\Config parselést és writeolást implementál. Az apache2 szerver parsere az előzőleg tárgyalt programokhoz képest bonyolultabb, ezért from scratch terveztem a parsert és a writert. Ehhez az apache2 config syntax dokumentációját vettem segítségül, amely megtalálható a \cite{apache_configuring} hivatkozás alatt.
	\\Az envvars nevű fájl szerkesztését implementáló class is ebben a modulban van.
\end{itemize}

\begin{center}
	Az apache \texttt{bootstrap} modul ezesetben is egyszerű felépítésű, a \texttt{\detokenize{config_handler}} pedig három osztályt tartalmaz:
	
	\hfill
\end{center}
	
\begin{tikzpicture} % [show background grid]
	\begin {class}[text width=0.45\textwidth]{apache}{0, 0}
		\attribute {+ \detokenize{server_impl -> }\textbf{apache\_server\_impl} module}
		\operation {+ \detokenize{is_apache_installed()}}
		\operation {+ \detokenize{install_apache()}}
		\operation {+ \detokenize{init_dirs()}}
	\end {class}
	
	\begin {class}[text width=0.45\textwidth]{\detokenize{apache_config_handler}}{7.6, 0}
		\attribute {- \detokenize{CR : string}}
		\attribute {- \detokenize{LF : string}}
		\attribute {- \detokenize{inspect -> }\textbf{inspect} module}
		\attribute {- \detokenize{general -> }\textbf{general} module}
		\attribute {+ \detokenize{apacheConfigHandler -> }\textbf{apacheConfigHandler} class}
		\attribute {+ \detokenize{apacheEnvvarsHandler -> }\textbf{apacheEnvvarsHandler} class}
		\operation {- \detokenize{parse_apache_config(linesInStr)}}
		\operation {- \detokenize{doPaddingWithBlockDeepness(blockDeepness)}}
		\operation {- \detokenize{formatDataAccordingQuoting(tbl)}}
		\operation {- \detokenize{write_apache_config(parsedLines)}}
		\operation {- \detokenize{parse_envvar_args_from_line(line)}}
		\operation {- \detokenize{escape_magic(s)}}
	\end {class}
	
	\begin {class}[text width=0.44\textwidth]{\detokenize{apacheConfigHandler}}{0, -5}
		\operation {+ \detokenize{new(linesInStr, paramToLine)}}
		\operation {+ \detokenize{getParsedLines()}}
		\operation {+ \detokenize{getParamsToIdx()}}
		\operation {+ \detokenize{insertNewData(dataTbl, pos)}}
		\operation {+ \detokenize{deleteData(pos)} - jelenleg nem használt}
		\operation {+ \detokenize{toString()}}
	\end {class}
	
	\begin {class}[text width=0.44\textwidth]{\detokenize{apacheEnvvarsHandler}}{5, -11}
		\operation {+ \detokenize{new(linesInStr)}}
		\operation {+ \detokenize{getArgs()}}
		\operation {+ \detokenize{toString()}}
	\end {class}
	
	\unidirectionalAssociation{apache_config_handler}{}{}{apacheConfigHandler}
	\unidirectionalAssociation{apache_config_handler}{}{}{apacheEnvvarsHandler}
\end {tikzpicture}

\pagebreak

\begin{center}
	Az apache esetében a \texttt{server\_impl} modul hasonló bonyolultságú az \texttt{nginx\_server\_impl} modulhoz:
	
	\hfill
	
	\begin{tikzpicture} % [show background grid]
		\begin {class}[text width=0.45\textwidth]{\detokenize{apache_server_impl}}{0, 0}
			\attribute {- \detokenize{os -> }\textbf{os} module}
			\attribute {- \detokenize{packageManager -> }\textbf{\detokenize{apt_packages}} module}
			\attribute {- \detokenize{linux -> }\textbf{linux} module}
			\attribute {- \detokenize{general -> }\textbf{general} module}
			\attribute {- \detokenize{apacheConfigHandlerModule -> }\textbf{\detokenize{config_handler}} module}
			\attribute {- \detokenize{inspect -> }\textbf{inspect} module}
			\attribute {+ \detokenize{apache_user : string}}
			\attribute {+ \detokenize{apache_user_comment : string}}
			\attribute {+ \detokenize{apache_user_shell : string}}
			\attribute {+ \detokenize{base_dir : string}}
			\attribute {- \detokenize{sampleConfigForWebsite : string}}
			\operation {+ \detokenize{formatPathInsideBasedir(path)}}
			\operation {+ \detokenize{check_apache_user_existence()}}
			\operation {+ \detokenize{create_apache_user(homeDir)}}
			\operation {+ \detokenize{update_existing_apache_user()}}
			\operation {+ \detokenize{get_apache_home_dir()}}
			\operation {+ \detokenize{get_apache_master_config_}\\\detokenize{path_from_daemon()}}
			\operation {+ \detokenize{init_dirs()}}
			\operation {+ \detokenize{initialize_server()}}
			\operation {+ \detokenize{create_new_website(websiteUrl)}}
			\operation {+ \detokenize{delete_website(websiteUrl)}}
			\operation {+ \detokenize{get_current_available_websites()}}
			\operation {+ \detokenize{init_ssl_for_website(webUrl, certDetails)}}
		\end {class}
	\end {tikzpicture}
\end{center}
	
\Section{certbot modul felépítése, feladata}

A certbot csak egy modult foglal magában. Működéséhez több modult is felhasznál. A modulban megtalálható a certbot telepítése snapd-n keresztül, symlink létrehozás, SSL certificate létrehozás HTTP-01 challenge és DNS challenge segítségével, továbbá be is konfigurálja az adott webszervereket a használathoz.

Habár a certbot akár az \texttt{apt} package managerrel is felrakható, az \texttt{apt-get install certbot} parancs segítségével, maga a program weboldalán a leírás szerint snapd-vel telepítik. \cite{certbot_install_debian}

A különbség főképp a kettő között az, hogy az \texttt{apt} package manager deb csomagokkal dolgozik, a \texttt{snap} pedig teljes archívumokkal.

Az \texttt{apt} esetében a függőségek külön rakódnak fel a csomagok feltelepítésekor, és a \texttt{deb} archívumok csak a csomagokat tartalmazzák. Ebben az esetben a frissítések késhetnek, mivel több szervezet, személy is átnézi a frissítés tartalmát.

A \texttt{snapd} esetében egy teljes archívumot tölt le a rendszer, ebben benne van a csomag összes függősége pont azzal a verzióval, amellyel a csomag fejlesztői szerették volna. Ez az archívum egy "zárt térbe" kerül kicsomagolásra, és limitált hozzáférése van magához a rendszerhez (hasonlóan a Docker-alkalmazásokhoz). Frissítéskor a fejlesztő adja ki a frissítéseket, nem validálják őket külön, ezáltal gyorsabban eljut a userekhez. \cite{snap_vs_apt}

\begin{center}
	A \texttt{certbot} modul UML diagramja:
	
	\hfill
	
	\begin{tikzpicture} % [show background grid]
		\begin {class}[text width=0.6\textwidth]{\detokenize{certbot}}{0, 0}
			\attribute {- \detokenize{os -> }\textbf{os} module}
			\attribute {- \detokenize{aptPackageManager -> }\textbf{\detokenize{apt_packages}} module}
			\attribute {- \detokenize{snapPackageManager -> }\textbf{\detokenize{snapd}} module}
			\attribute {- \detokenize{linux -> }\textbf{linux} module}
			\attribute {- \detokenize{general -> }\textbf{general} module}
			\attribute {- \detokenize{inspect -> }\textbf{\detokenize{inspect}} module}
			\attribute {- \detokenize{nginx -> }\textbf{\detokenize{nginx}} module}
			\attribute {- \detokenize{apache -> }\textbf{\detokenize{apache}} module}
			\attribute {- \detokenize{certFileName : string}}
			\attribute {- \detokenize{keyFileName : string}}
			\attribute {- \detokenize{dhParamFileName : string}}
			\attribute {- \detokenize{dryRunStr : string} - kizárólag debugra}
			\operation {- \detokenize{sleep(n)}}
			\operation {+ \detokenize{is_certbot_installed()}}
			\operation {+ \detokenize{create_certbot_symlink()}}
			\operation {+ \detokenize{install_certbot()}}
			\operation {+ \detokenize{get_cert_datas(domain)}}
			\operation {+ \detokenize{try_ssl_certification_creation(method, domain, webserverType)}}
		\end {class}
	\end {tikzpicture}
\end{center}

\Section{snapd modul felépítése, feladata}

A snapd modul egyszerű felépítésű. Feladata maga a snapd feltelepítése az \texttt{apt} package manager segítségével, továbbá csomagok telepítése, és a már feltelepített csomagok meglétének ellenőrzése.

\begin{center}
	A \texttt{snapd} modul UML diagramja:
	
	\hfill
	
	\begin{tikzpicture} % [show background grid]
		\begin {class}[text width=0.6\textwidth]{\detokenize{snapd}}{0, 0}
			\attribute {- \detokenize{os -> }\textbf{os} module}
			\attribute {- \detokenize{linux -> }\textbf{linux} module}
			\attribute {- \detokenize{aptPackageManager -> }\textbf{\detokenize{apt_packages}} module}
			\operation {+ \detokenize{isSnapdInstalled()}}
			\operation {+ \detokenize{installSnapd()}}
			\operation {+ \detokenize{is_package_installed(packageName)}}
			\operation {+ \detokenize{install_package(packageName, classic)}}
		\end {class}
	\end {tikzpicture}
\end{center}

\pagebreak

\Section{iptables modul felépítése, feladata}

Ez a modul kezeli a tűzfalszabályokat az \texttt{iptables} frontend segítségével. Funkcionalitása maga az iptables telepítése, a portok nyitása/zárása, kimenő/bemenő csomagok szűrése, továbbá a NAT előkészítése az OpenVPN szerver számára.

\begin{center}
	Az \texttt{iptables} modul UML diagramja:
	
	\hfill
	
	\begin{tikzpicture} % [show background grid]
		\begin {class}[text width=0.9\textwidth]{\detokenize{iptables}}{0, 0}
			\attribute {- \detokenize{os -> }\textbf{os} module}
			\attribute {- \detokenize{packageManager -> }\textbf{apt\_packages} module}
			\attribute {- \detokenize{general -> }\textbf{general} module}
			\attribute {- \detokenize{linux -> }\textbf{linux} module}
			\attribute {- \detokenize{inspect -> }\textbf{\detokenize{inspect}} module}
			\attribute {- \detokenize{iptablesAliases : table}}
			\operation {+ \detokenize{is_iptables_installed()}}
			\operation {+ \detokenize{install_iptables()}}
			\operation {+ \detokenize{get_current_ssh_ports()}}
			\operation {+ \detokenize{parse_current_rules()}}
			\operation {+ \detokenize{get_open_ports(interface)}}
			\operation {+ \detokenize{close_port(interface, protocol, dport, fromIP)}}
			\operation {+ \detokenize{open_port(interface, protocol, dport, fromIP)}}
			\operation {+ \detokenize{allow_outgoing_new_connection(interface, protocol, dip, dport)}}
			\operation {+ \detokenize{check_if_inbound_packets_are_being_filtered_already(interface, protocol)}}
			\operation {+ \detokenize{tog_only_allow_accepted_packets_inbound(toggle, interface, protocol)}}
			\operation {+ \detokenize{check_if_outbound_packets_are_being_filtered_already(interface, protocol)}}
			\operation {+ \detokenize{tog_only_allow_accepted_packets_outbound(toggle, interface, protocol)}}
			\operation {+ \detokenize{init_nat_for_openvpn(mainInterface, tunnelInterface, openvpnSubnet)}}
			\operation {+ \detokenize{iptables_to_string()}}
			\operation {+ \detokenize{init_module()}}
		\end {class}
	\end {tikzpicture}
\end{center}