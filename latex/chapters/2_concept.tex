\Chapter{Koncepció}

\Section{Felhasznált programnyelv}
A program elkészítéséhez a felhasznált programnyelv a Lua, tesztelésre az 5.3.5 verzió lett használva.

A Lua egy könnyű, magas szintű programozási nyelv, amelyet főképp a könnyű beágyazhatóság jegyében fejlesztettek ki. Cross-platform, mivel implementációja Ansi C-ben íródott. Saját virtuális géppel és bytecode formátummal rendelkezik.
A programok futtatás előtt bytecode-ra fordítódnak át, majd úgy kerül átadásra az interpreternek. Mi magunk is lefordíthatjuk a Lua forráskódunkat a luac bináris segítségével.
Ezt akkor szokták megtenni, ha fel akarják gyorsítani a program futását, vagy ha nem szeretnék, hogy idegenek ismerjék a forráskódot.

Támogat többféle programozási paradigmát is, azonban ezek nincsenek előre implementálva, viszont a nyelv lehetőséget ad arra, hogy implementáljuk őket. Például öröklődést, osztályokat metatáblák használatával tudunk implementálni. \cite{ooplua}

Széleskörű C API található felhasználásához, azonban nem csak erre a nyelvre korlátozódik az API használatának lehetősége, többféle wrapper is készült a C API-hoz, a legtöbb azonban a C++ nyelvhez készült (például sol).

Készült hozzá Just-In-Time Compiler is LuaJIT néven, amely alapjaiban is gyorsabb, mint a Lua. A LuaJIT bytecode formátuma teljesen más, mint a sima Lua-é, gyorsabb az instrukciók dekódolása. A virtuális gépe is közvetlenül Assembly-ben íródott. \cite {luajit}

Csomagkezelő is készült hozzá, amelyet LuaRocksnak hívnak, hasonló a funkcionalitása a NodeJS npm csomagkezelőjéhez.

Előszeretettel használják a játékfejlesztők is, több neves játékban is előfordul, például World of Warcraft, PayDay 2, Saints Row széria, Crysis. \cite{usageoflua}

\Section{Programtól elvárt működés}

Az OpenVPN implementációt a program tudja kezelni, telepíteni. A telepítést az apt-get beépített segédprogrammal végzi. Kezelni tudja az alábbi dolgokat:
\begin{itemize}
 \item authentikáció módjának módosítását (kulcsalapú; felhasználónév/jelszó alapú)
 \item naplózási beállításokat tud módosítani
 \item WebAdmin felület tud konfigurálni Apache2/Nginx+PHP+MySQL-lel
 \item kliensek létrehozása (felhasználónév/jelszó alap, vagy kulcs), törlése (kulcs esetén revoke)
 \item kliensek számára személyre szabott .ovpn config generálás
 \item init.d beállítások, az eredeti daemon beállítása automatikus indításra
 \item külön user létrehozása a szerver futtatására
\end{itemize}

Apache2/Nginx webszervert is tud kezelni a program, ebbe beletartozik:

\begin{itemize}
	\item telepítés apt-gettel
	\item honlap hozzáadása külön directoryval, felhasználóval
	\item honlap törlése
	\item reverse proxy kezelése
	\item SSL certificate kezelés többféle segédeszközzel, Let's Encrypton belül dehydrated és certbot segítségével
\end{itemize}

Tűzfal gyanánt az iptables nevű beépített Linux segédprogramot tudja kezelni többféle aspektusból:
\begin{itemize}
	\item telepítés apt-gettel ha nincs fent
	\item IPv4 és IPv6-ot egyaránt támogat
	\item Port nyitás, zárás
	\item Bizonyos portra csak bizonyos IP-ről csatlakozás engedélyezése
	\item Bizonyos IP cím felé csak bizonyos kimeneti portok felé kimenő kapcsolat engedélyezése
	\item Rate limit a portokra a hashlimit modul segítségével
	\item OpenVPN szerverhez köthető NAT Forwardot csinál, ha minden forgalmat a szerverre szeretnénk irányítani
	\item A fenti szabályok korlátozhatóak az egyes network interfacekra
	\item Szabályellenőrzés, például ha engedélyezünk egy portot, és nincsenek letiltva a nem engedélyezett bejövő kapcsolatok, akkor figyelmeztetés
	\item Ki-be kapcsolható kapcsolók:
		\begin{itemize}
			\item Bejövő forgalom csak az engedélyezettek közül jöhet be
			\item Kimenő forgalom csak az engedélyezettek felé mehet ki
		\end{itemize}
	\item Hozzáadott, törölt szabályok kiírása tanítási célból
\end{itemize}

\Section{A fejezet célja}

Ez a fejezet még nem a saját eredményekkel foglalkozik, hanem bemutatja, mi a problémakör, milyen módszerekkel, milyeneredményeket sikerült elérni eddig másoknak.

A hivatkozások jelentős része ehhez a fejezethez szokott kötődni.
(Egy hivatkozás például így néz ki \cite{coombs1987markup}.)
Itt lehet bemutatni a hasonló alkalmazásokat.

\Section{Tartalom és felépítés}

A fejezet tartalma témától függően változhat. Az alábbiakat attól függően különböző arányban tartalmazhatják.
\begin{itemize}
\item Irodalomkutatás. Amennyiben a dolgozat egy módszer kidolgozására, kifejlesztésére irányul, akkor itt lehet részletesen végignézni (módszertani vagy időrendi bontásban), hogy az eddigiekben milyen eredmények születtek a témakörben.
\item Technológia. Mivel jellemzően kutatásról vagy szoftverfejlesztésről van szó, ezért annak a jellemző elemeit, technikai részleteit itt kell bemutatni.
Ez tehát egy módszeres bevezetés ahhoz, hogy ha valaki nem jártas a témakörben, akkor tudja, hogy a dolgozat milyen aktuálisan elérhető eredményeket, eszközöket használt fel.
\item Piackutatás. Bizonyos témáknál új termék vagy szolgáltatás kifejlesztése a cél.
Ekkor érdemes annak alaposan utánanézni, hogy aktuálisan milyen eszközök érhetők el a piacon.
Ez szoftverek esetében a hasonló alkalmazások bemutatását, táblázatos formában történő összehasonlítását jelentheti.
Szerepelhetnek képek és észrevételek a viszonyításként bemutatott alkalmazásokhoz.
\item Követelmény specifikáció. Külön szakaszban érdemes részletesen kitérni az elkészítendő alkalmazással kapcsolatos követelményekre.
Ehhez tartozhatnak forgatókönyvek (\textit{scenario}-k).
A szemléletesség kedvéért lehet hozzájuk képernyőkép vázlatokat is készíteni, vagy a használati eseteket más módon szemléltetni.
\end{itemize}

\Section{Amit csak említés szintjén érdemes szerepeltetni}

Az olvasóról annyit feltételezhetünk, hogy programozásban valamilyen szinten járatos, és a matematikai alapfogalmakkal sem ebben a dolgozatban kell megismertetni.
A speciális eszközök, programozási nyelvek, matematikai módszerekk és jelölések persze jó, hogy ha említésre kerülnek, de nem kell nagyon belemenni a közismertnek tekinthető dolgokba.
